\documentclass{article}
\usepackage[utf8]{inputenc}
\usepackage{graphicx}
\usepackage{listings}
\usepackage{xcolor}
\usepackage{amsmath}
\usepackage{hyperref}

% Configurações para o código Python
\definecolor{codegreen}{rgb}{0,0.6,0}
\definecolor{codegray}{rgb}{0.5,0.5,0.5}
\definecolor{codepurple}{rgb}{0.58,0,0.82}
\definecolor{backcolour}{rgb}{0.95,0.95,0.92}

\lstdefinestyle{mystyle}{
    backgroundcolor=\color{backcolour},   
    commentstyle=\color{codegreen},
    keywordstyle=\color{magenta},
    numberstyle=\tiny\color{codegray},
    stringstyle=\color{codepurple},
    basicstyle=\ttfamily\footnotesize,
    breakatwhitespace=false,         
    breaklines=true,                 
    captionpos=b,                    
    keepspaces=true,                 
    numbers=left,                    
    numbersep=5pt,                  
    showspaces=false,                
    showstringspaces=false,
    showtabs=false,                  
    tabsize=2
}

\lstset{style=mystyle}

\title{Visualização de um Modelo Atômico em 3D}
\author{Seu Nome}
\date{\today}

\begin{document}

\maketitle

\section{Introdução}
Este documento descreve a implementação de um script em Python para visualizar um modelo atômico em 3D com animação. O script utiliza as bibliotecas \texttt{matplotlib} e \texttt{numpy} para criar uma representação animada de elétrons orbitando um núcleo.

\section{Código Python}
Abaixo está o código Python completo para a visualização do modelo atômico em 3D.

\begin{lstlisting}[language=Python, caption={Script Python para visualização de um modelo atômico em 3D}]
import numpy as np
import matplotlib.pyplot as plt
from mpl_toolkits.mplot3d import Axes3D
import matplotlib.animation as animation

# Constants
ELECTRON_COLOR = "blue"
NUCLEUS_COLOR = "red"
ELECTRON_SIZE = 50
NUCLEUS_SIZE = 100
ORBIT_RADIUS = 5
NUM_ELECTRONS = 8
FRAMES = 100

# Create a figure and 3D axis
fig = plt.figure()
ax = fig.add_subplot(111, projection="3d")

# Set axis limits
ax.set_xlim([-ORBIT_RADIUS, ORBIT_RADIUS])
ax.set_ylim([-ORBIT_RADIUS, ORBIT_RADIUS])
ax.set_zlim([-ORBIT_RADIUS, ORBIT_RADIUS])

# Hide grid and axes for better visualization
ax.grid(False)
ax.set_xticks([])
ax.set_yticks([])
ax.set_zticks([])

# Draw the nucleus (central atom)
nucleus = ax.scatter([0], [0], [0], color=NUCLEUS_COLOR, s=NUCLEUS_SIZE)

# Initialize electron positions
electrons = ax.scatter([], [], [], color=ELECTRON_COLOR, s=ELECTRON_SIZE)

# Function to update electron positions for animation
def update(frame):
    # Calculate electron positions in circular orbits
    angles = np.linspace(0, 2 * np.pi, NUM_ELECTRONS, endpoint=False)
    x = ORBIT_RADIUS * np.cos(angles + frame * 0.1)
    y = ORBIT_RADIUS * np.sin(angles + frame * 0.1)
    z = np.zeros_like(x)

    # Update electron positions
    electrons._offsets3d = (x, y, z)
    return electrons,

# Create the animation
ani = animation.FuncAnimation(
    fig, update, frames=FRAMES, interval=50, blit=True
)

# Display the animation
plt.show()
\end{lstlisting}

\section{Explicação do Código}
O código Python acima cria uma visualização 3D de um modelo atômico com as seguintes características:

\begin{itemize}
    \item \textbf{Núcleo}: Representado por um ponto central de cor vermelha.
    \item \textbf{Elétrons}: Representados por pontos azuis que orbitam o núcleo em trajetórias circulares.
    \item \textbf{Animação}: Utiliza a função \texttt{FuncAnimation} da biblioteca \texttt{matplotlib.animation} para criar uma animação suave dos elétrons em movimento.
\end{itemize}

\section{Requisitos}
Para executar o script, é necessário instalar as seguintes bibliotecas Python:

\begin{verbatim}
pip install numpy matplotlib
\end{verbatim}

\section{Resultado}
Ao executar o script, uma janela será aberta exibindo a animação 3D do modelo atômico, com elétrons orbitando o núcleo em tempo real.

\section{Conclusão}
Este projeto demonstra como utilizar Python e bibliotecas gráficas para criar visualizações científicas interativas. O código pode ser facilmente adaptado para representar outros modelos físicos ou químicos.

\end{document}
